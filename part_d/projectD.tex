\documentclass[a4paper]{report}
\usepackage{hyperref}
\usepackage{lastpage}
\usepackage{fancyhdr}
\usepackage{lineno}
\usepackage{listings}
\usepackage{german}
\usepackage[utf8]{inputenc}
\usepackage{amssymb}
\usepackage{graphicx}
%\newcommand{\genasso}[2]{\begin{minipage}{0.7\textwidth}\begin{normalsize}\begin{flushleft}\textbf{{#1}}\end{flushleft}\end{normalsize}\vspace{-1cm}\begin{flushleft}\begin{small}{#2}\end{small}\end{flushleft}\end{minipage}\\\vspace{0.2cm}}
\pagenumbering{arabic}

\pagestyle{fancy} 
\newcommand{\frontmatter}{\clearpage \cfoot{\thepage\ }
\setcounter{page}{1}
\pagenumbering{Roman}}
\newcommand{\mainmatter}{\clearpage \lhead{\myAuth} \rhead{\myDate} \cfoot{} \rfoot{\thepage\ of \pageref{LastPage}}
\setcounter{page}{1}
\pagenumbering{arabic}}
\newcommand{\backmatter}{\clearpage \rfoot{\thepage\ }
\setcounter{page}{1}
\pagenumbering{alph}}


\newcommand{\makemytitlepage}{\begin{titlepage}
    \begin{center}
        \vspace*{0.8cm}
        
        \Huge
        \textbf{\myTitle}
        
        \vspace{1.5cm}
        
        \Large
        \myAuthor

        \vspace{1.8cm}

        %\begin{large}\textbf{Abstract:} \myAbstract \end{large}
        \includegraphics[width=6cm]{./IM.jpg}  
        
        \vfill
        
        \huge
        \myAsso
        
        \vspace{1.3cm}
        
        \Large

        %\myDate
        \today
        
    \end{center}
\end{titlepage}}
\newcommand{\myAuth}{Team: *Iron Man*\\B. Pohl, K. Trogant, R. Enseleit, D. Hebecker}
\newcommand{\myAuthor}{Birgit Pohl 574353 (MO. 9-11)\\Kevin Trogant 572451 (Mo. 15-17)\\Ronja Enseleit 572404 (Mo. 15-17)\\Dustin Hebecker 571271 (MO. 9-11)}
\newcommand{\myAsso}{Group: *Iron Man*}
\newcommand{\myDate}{\today}

%%%%%%%%%%%%%%%%%%%%%%%%%%%%%%%%
%%Change Title !!!!!!!!!!!!!!!!!
%%%%%%%%%%%%%%%%%%%%%%%%%%%%%%%%
\newcommand{\myTitle}{Exercise Sheet D}

\begin{document}
\frontmatter
\makemytitlepage
\mainmatter

%%%%%%%%%%%%%%%%%%%%%%%%%%%%%%%%%%%%%%%%%%%%%%%%%%%%%%%%%%
%% Only modify below here  and change myTitle!!!!!!!!!!!!!
%%%%%%%%%%%%%%%%%%%%%%%%%%%%%%%%%%%%%%%%%%%%%%%%%%%%%%%%%%
\section*{Aufgabe 1}
\subsection*{Systemvorraussetzungen}
\begin{enumerate}
\item Stellt sicher das Python in mindestens version 3.4 installiert ist

\item Pip(3) wird empfohlen zur Installation im folgenden empfohlen ist jedoch nicht notwending.

\item Django (getestet mit Version 1.10.5)

\item Möglicherweise werden weitere Pakete benötigt. Im unerwarteten Falle dessen, erhalten sie hierzu entsprechende Fehlermeldungen beim Aufrufen des folgenden Befehl.

\item Zum aktivieren des Entwicklungsservers in das \url{PROJECT/pizza} Verzeichnis gehen und mittels
\begin{lstlisting}
python manage.py runserver
\end{lstlisting}
den Server starten. (Erreichbar unter 127.0.0.1:8000)
\end{enumerate}

\subsection*{Nutzung (Kunde/Angestellter/Admin)}

In die Kundensicht gelangt man über die Adresse: \url{http://127.0.0.1:8000/} von wo aus die Bedinung intuitiv sein sollte.\\ \\

In die Angestelltensicht gelangt man über die Adresse: \url{http://127.0.0.1:8000/staff/open_orders}, welche noch intuitiver ist und zur Zeit nicht Passwort geschützt ist.\\ \\

Da Pizzen statisch sein sollen kann der Angesetllte die Rezepte nicht ändern. Dies ist jedoch dem Admin über die Adresse \url{http://127.0.0.1:8000/admin/} erlaubt. Da dies jedoch nicht mehr Teil der Aufgabenstellung ist bleibt es dem Leser selber überlassen das nötige Passwort zu finden ;-).


\subsection*{Programmstruktur}
Die wesentlichen Komponenten befinden sich im Ordner \url{PROJECT/pizza/}, hier findet sich die Datei manage.py zum Aufrufen des Servers und die Datenbank.\\
Im Unterordner  \url{PROJECT/pizza/orders/} findet sich der Haupteil des Codes.\\
In der Datei Models.py wird die Datenbankstruktur durch ein Pythoninterface festgelegt.\\
In der Datei views.py werden die einzelnen Seiteninhalte berechnet, zur gestalltung finden sich dann dann die entsprechenden Templates im ordner  \url{PROJECT/pizza/orders/templates/} und einige utility Funktionen in utils.py\\
Alle Blackboxtesting Funktionen finden sich in der datei tests.py.

\newpage
\section*{Aufgabe 2}
Zum testen wird der Befehl
\begin{lstlisting}
python manage.py test
\end{lstlisting}
aufgerufen. Die Test selber können in \url{PROJECT/pizza/orders/tests.py} gefunden werden.

\subsection*{Blackbox Testing}
\begin{enumerate}
\item Download Firefox 51.0.1
\item Add Selenium IDE as Add-on to Firefox, 2.9.1 \url{(https://addons.mozilla.org/de/firefox/addon/selenium-ide/)}
\item Add Selenium IDE: Flow Control (Due to Errors with the FF 51) \url{(https://addons.mozilla.org/en-US/firefox/addon/flow-control/?src=dp-dl-othersby)}
\item Reopen Firefox
\item Open Selenium in Tools (EN)/Extras(DE) -> Selenium IDE
\item Go to File(EN)/Datei(DE) -> \glqq Open Test Suite\grqq to import the test suite
\item Click on \glqq Play Entire Test Suite\grqq button
\item Watch the magic happening
\end{enumerate}


\end{document}
